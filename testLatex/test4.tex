\documentclass{article}
\usepackage{amsmath}

\begin{document}
\title{this is the title}
\author{author, Affilication}
\date{}

\maketitle

\begin{abstract}
this is sample abstract
\end{abstract}

\section{Introduction}
this is sample section

\begin{enumerate}
    \item the labels consists of sequential numbers
    \item the numbers starts at 1 with every call to the enumerate environment
\end{enumerate}

\begin{itemize}
    \item the individual entries are indicated with a block dot, a so-called bullet
    \item the text in the entries may be of any length
\end{itemize}

\begin{description}
    \item[Purpose:] this environment is appropriate when a number of works or
                    expressions are to be defined. This environment is appropriate when a number of words or expressions are to be defined.
    \item[Example:] It may also be used as an author list in the bibliography
\end{description}

% nested list
\begin{itemize}
    \item The {\tt itemize} label at the first level is a bullet.
    \begin{enumerate}
        \item The numbering is with Arabic numerals since this is ...
        \begin{itemize}
            \item This is the third level of the nesting, but the ...
                \begin{enumerate}
                    \item And this is the fourth level of the overall ...
                    \item Thus the numbering is with lower case letters ...
                \end{enumerate}
            \item The label at this level is a long dash.
        \end{itemize}
        \item Every list should contain at least two points.
    \end{enumerate}
    \item Blank lines ahead of an ...
\end{itemize}

\subsection{Subsection}
This is sample subsection

% box exercise
% it seems that pdflatex and indiaUserGroup use different version of Latex
\makebox[5cm][c] {some word}        \par
\framebox[5cm][c]{some words}

\framebox{A few words of advice}
\framebox[\width+4mm][s]{A few words of advice}
\framebox[1.5\width]{A few word of advice}

\fbox{Text in a box}
\setlength\fboxrule{2pt}\setlength\fboxsep{2mm}
\fbox{Text in a box}

% baseline \arisebox{lex}{upward} baseline
% \raisebox{-lex}{downword} baseline
% TODO: ??
% baseline \arisebox{10pt}{upward} baseline \raisebox{-10pt}{downword} baseline

% \doturn doesn't work with pdflatex
$x_1$ \doturn{\fbox{badthing}}
$x_2$ \doturn{\raisebox{\depth}{\fbox{bad thing}}}
$x_3$ \doturn{\raisebox{-\height}{\fbox{Bad thing}}} 
$x_4$

\parbox{.3\linewidth}
	{this is the content of the left-most parbox. }
\hfill CURRENT LINE \hfill
\parbox{.3\linewidth}
	{this is the right-most parbox. Note that the typeset text looks sloppy because 
	\LaTeX{} cannot nicely balance the material in these narrow columns. }

\subsubsection{Subsubsection}
this is sample subsubsection

\paragraph{paragraph}
this is sample paragraph.

\begin{thebibliography}{00}

\bibitem{1} this is sample bibitem one.
\bibitem{2} this is sample bibitem two.
\bibitem{3} this is sample bibitem three.

\end{thebibliography}

\end{document}
